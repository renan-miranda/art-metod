\documentclass[12pt]{article}
\usepackage{sbc-template}
\usepackage{graphicx,url}
\usepackage[brazil]{babel}
% \usepackage[latin1]{inputenc}
\usepackage[utf8]{inputenc}

\sloppy

\title{Diferentes visões sobre o aquecimento global em função da emissão de CO2}

\author{Renan França Miranda\inst{1}}


\address{Departamento de Informática -- Universidade Federal do Paraná
  (UFPR)\\
  \email{rfm08@inf.ufpr.br}
}

\begin{document}

\maketitle

\begin{abstract}
Global warming has emerged in many discussions today. On
this subject, there are different positions on main causes and 
impact on our lives. One reason is the most talked about issue
carbon \textit{(CO2)} in our atmosphere. Therefore, in this article
we will see different positions on global warming in terms of CO2 emissions.
It will have an introduction, a display and analysis of some points of view
and a conclusion on the subject.
\end{abstract}

% Comando \keywords dá erro de compilação

\textit{\textbf{Keywords}: Global warming, carbon dioxide, CO2, discussion}

\begin{resumo}
O aquecimento global vem surgindo em muitas discussões atualmente. Sobre
esse assunto, temos diferentes posições sobre suas principais causas e seu
impacto em nossas vidas. Um dos motivos mais comentados é sobre a emissão de
gás carbônico \textit{(CO2)} na nossa atmosfera. Por isso, neste artigo
veremos diferentes posições sobre o aquecimento global em termos da emissão de CO2.
Nela teremos uma introdução, uma exibição e análise de alguns pontos de vista e uma conclusão sobre o assunto.
\end{resumo}

\textit{\textbf{Palavras-chave}: Aquecimento Global, gás carbônico, CO2, discussão}

\section{Introdução}

Ao longo dos séculos, o ser humano tem feito várias evoluções, entre
elas a teconológica. Sempre buscamos a melhoria de serviços, para que
a faciliade no dia-a-dia seja maior.

O processo de automatização de serviços teve um grande crescimento após
a revolução industrial, onde nela tivemos a superação da máquina frente
ao homem, mostrando que era capaz de subustuí-lo em grande parte das
operações. Com isso, o ser humano investiu pesado e criou novas alternativas
para serviços utilizando-se de máquinas.

Com essa mudança, a máquina começou a ser implantada em grandes e pequenos
serviços, e sua utilização começou a ser maior. Como as máquinas se utilizam
de combustível para funcionar, houve também o aumento de emissão de gás
carbônico na atmosfera, o que pode ser prejudicial para o nosso meio
ambiente.

Com as constantes mudanças climáticas, começou a discussão sobre o ser
humano e sua participação no aquecimento global com a grande emissão de gás
carbônico na nossa atmosfera. Sobre isso iremos ver pontos diferentes de
visão sobre esse assunto atráves de opiniões de especialistas no assunto.

\section{Visões sobre a emissão de CO2}

\subsection{Visão negativa}
Para o físico Paulo Artaxo, integrante do IPCC \textit{(Intergovernmental
Panel on Climate Change)}, que esteve presente no debate sobre aquecimento
global da 62ª reunião da SBPC \textit{(Sociedade Brasileira para o Progresso
da Ciência)} ~\cite{Medeiros}, o ser humano que é responsável pela elevação na
temperatura através da grande emissão de CO2 desde a revolução industrial.
Além de exibir dados relevantes, como que a concentração de gases-estufa na
atmosfera é extremamente elevada, cita também que 12 dos últimos 14 anos são
os mais quentes da história, e que é visível essa mudança remetendo à alta
incidência de fenômenos naturais que estão acontecendo. Para ele, em algumas
áreas, estaria havendo mais chuvas, enquanto que em outras, a seca é mais
frequente. Por fim, acha que é necessário mudanças na matriz energética
mundial, hoje dominada pela combustão de fontes fósseis, e alterações no
padrão de consumo global.

Quem também tem a mesma visão são um grupo de cientistas canadenses
~\cite{Reuters}, que após uma série de simulações feitas, a emissão de CO2 feita
até hoje vai continuar a contribuir para o aquecimento global durante
séculos. Seus estudos deram vários resultados, como que nos próximos mil anos, a
temperatura média do oceano ao redor da Antártica pode aumentar em até 5
graus Celsius, provocando o colapso do manto de gelo da Antártica Ocidental,
impactos do clima reduzirão a umidade em partes do norte da África em até
30\%, mas também dizem que muitas das consequências negativas no Hemisfério
Norte, como a redução de gelo no Ártico, são reversíveis, dizendo que os
esforços para a diminuição de emissão de gases não são um desperdício de
esforço e dinheiro.

\subsection{Visão positiva}

Para o geógrafo Ricardo Felício, que também esteve presente no debate da
SBPC ~\cite{Medeiros}, o problema não é a quantidade de emissão de gás, mas
sim o papel do homem nela, apresentando falhas, como mostrar fotos do
medidores estçao próximos a fornos, tetos de edifícios e até churrasqueiras,
e erros de algoritmos utilizados na simulações. Também expõe a natureza,
dizendo que as mudanças climáticas são fenômenos naturais e que nós
liberamos menos CO2 que os insetos. Mas o que importa para ele é essa causa
ambiental estar sendo usada como uma nova relação política para manter o
poder mundial, onde se troca antigas relações comerciais por outras,
travestidas de ecológicas. Com isso, grandes países estariam impondo acordos
comerciais ou produtos que beneficiam suas indústrias detentoras de
tecnologias ''limpas''. Para concluir, ele acha que é necessário o Brasil
ficar de olho nos novos acordos para escapar desse novo imperialismo.

O biólogo Fernando Reinach, em seu artigo ~\cite{Fernando}, defende o gás
carbônico, dizendo que quem põe a culpa na emissão de CO2 no aquecimento
global é desinformado ou manipulado, pois qualquer ser vivo na Terra emite
ou consume CO2 para sua existência e acabar com a emissão seria acabar com a
vida no planeta. Também explica o que aconteceu com o DNA, que em
campanhas contra a biotecnologia o DNA foi tão difamado que várias pessoas
não queriam comer alimentos que continham DNA, sendo que todo dia comemos
enormes quantidades de DNA. Na verdade substituíram o medo de agrotóxicos
pelo medo de DNA. Para ele, a culpa não está na molécula, e sim na queima de
combustível fóssil 

\section{Conclusão}
Através desses pontos de vista, pode-se concluir que para vários segmentos
de estudos, a ótica do problema é diferente, pois para quem está em
organizações que defendem a causa, o aquecimento é causado apenas por nós,
esquecendo-se que as pessoas não são maiores que tudo, sendo apenas parte de
um sistema maior. Nesse caso, é possível ver que a preocupação é tanta, mas
idéias para se evitar problemas maiores são poucas, pois as grandes
indústrias pensam que mudar sua ideologia de produção custa muito tempo e
dinheiro, e por enquanto elas pensam somente no lucro.

Já para o outro lado, a visão é que o problema não é nosso, pois vários
dos fenômenos naturais que acontecem não são culpa nossa, e o que está
acontecendo na verdade é uma manipulação da informação que acontece para se
conseguir mais e mais poder através de acordos onde se insere um valor
social muito forte, tentando assim juntar a população para essa causa. Essa
visão é importante, pois não sabemos com que intenção esses novos acordos
estão sendo feitos e para que finalidade exatamente.

De forma geral, é evidente que o problema de aquecimento global tende a
evoluir se não for feito nada para a redução de emissão de CO2, mas temos
que fazer de forma inteligente e cautelosa, para não existir manipulação por
alguma das partes.


% \subsection{Subsections}

% \section{Figures and Captions}\label{sec:figs}

%   Figure and table captions should be centered if less than one line
%   (Figure~\ref{fig:exampleFig1}), otherwise justified and indented by 0.8cm on
%   both margins, as shown in Figure~\ref{fig:exampleFig2}. The caption font must
%   be Helvetica, 10 point, boldface, with 6 points of space before and after each
%   caption.

% \begin{figure}[ht]
% \centering
% \includegraphics[width=.5\textwidth]{fig1.jpg}
% \caption{A typical figure}
% \label{fig:exampleFig1}
% \end{figure}

% \section{Referências}

\bibliographystyle{sbc}{}
\bibliography{sbc-template}

\end{document}
