\section{Introdução}

Ao longo dos séculos, o ser humano tem feito várias evoluções, entre
elas a teconológica. Sempre buscamos a melhoria de serviços, para que
a faciliade no dia-a-dia seja maior.

O processo de automatização de serviços teve um grande crescimento após
a revolução industrial, onde nela tivemos a superação da máquina frente
ao homem, mostrando que era capaz de subustuí-lo em grande parte das
operações. Com isso, o ser humano investiu pesado e criou novas alternativas
para serviços utilizando-se de máquinas.

Com essa mudança, a máquina começou a ser implantada em grandes e pequenos
serviços, e sua utilização começou a ser maior. Como as máquinas se utilizam
de combustível para funcionar, houve também o aumento de emissão de gás
carbônico na atmosfera, o que pode ser prejudicial para o nosso meio
ambiente.

Com as constantes mudanças climáticas, começou a discussão sobre o ser
humano e sua participação no aquecimento global com a grande emissão de gás
carbônico na nossa atmosfera. Sobre isso iremos ver pontos diferentes de
visão sobre esse assunto atráves de opiniões de especialistas no assunto.
