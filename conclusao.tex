\section{Conclusão}
Através desses pontos de vista, pode-se concluir que para vários segmentos
de estudos, a ótica do problema é diferente, pois para quem está em
organizações que defendem a causa, o aquecimento é causado apenas por nós,
esquecendo-se que as pessoas não são maiores que tudo, sendo apenas parte de
um sistema maior. Nesse caso, é possível ver que a preocupação é tanta, mas
idéias para se evitar problemas maiores são poucas, pois as grandes
indústrias pensam que mudar sua ideologia de produção custa muito tempo e
dinheiro, e por enquanto elas pensam somente no lucro.

Já para o outro lado, a visão é que o problema não é nosso, pois vários
dos fenômenos naturais que acontecem não são culpa nossa, e o que está
acontecendo na verdade é uma manipulação da informação que acontece para se
conseguir mais e mais poder através de acordos onde se insere um valor
social muito forte, tentando assim juntar a população para essa causa. Essa
visão é importante, pois não sabemos com que intenção esses novos acordos
estão sendo feitos e para que finalidade exatamente.

De forma geral, é evidente que o problema de aquecimento global tende a
evoluir se não for feito nada para a redução de emissão de CO2, mas temos
que fazer de forma inteligente e cautelosa, para não existir manipulação por
alguma das partes.
