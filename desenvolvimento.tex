\section{Visões sobre a emissão de CO2}

\subsection{Visão negativa}
Para o físico Paulo Artaxo, integrante do IPCC \textit{(Intergovernmental
Panel on Climate Change)}, que esteve presente no debate sobre aquecimento
global da 62ª reunião da SBPC \textit{(Sociedade Brasileira para o Progresso
da Ciência)} ~\cite{Medeiros}, o ser humano que é responsável pela elevação na
temperatura através da grande emissão de CO2 desde a revolução industrial.
Além de exibir dados relevantes, como que a concentração de gases-estufa na
atmosfera é extremamente elevada, cita também que 12 dos últimos 14 anos são
os mais quentes da história, e que é visível essa mudança remetendo à alta
incidência de fenômenos naturais que estão acontecendo. Para ele, em algumas
áreas, estaria havendo mais chuvas, enquanto que em outras, a seca é mais
frequente. Por fim, acha que é necessário mudanças na matriz energética
mundial, hoje dominada pela combustão de fontes fósseis, e alterações no
padrão de consumo global.

Quem também tem a mesma visão são um grupo de cientistas canadenses
~\cite{Reuters}, que após uma série de simulações feitas, a emissão de CO2 feita
até hoje vai continuar a contribuir para o aquecimento global durante
séculos. Seus estudos deram vários resultados, como que nos próximos mil anos, a
temperatura média do oceano ao redor da Antártica pode aumentar em até 5
graus Celsius, provocando o colapso do manto de gelo da Antártica Ocidental,
impactos do clima reduzirão a umidade em partes do norte da África em até
30\%, mas também dizem que muitas das consequências negativas no Hemisfério
Norte, como a redução de gelo no Ártico, são reversíveis, dizendo que os
esforços para a diminuição de emissão de gases não são um desperdício de
esforço e dinheiro.

\subsection{Visão positiva}

Para o geógrafo Ricardo Felício, que também esteve presente no debate da
SBPC ~\cite{Medeiros}, o problema não é a quantidade de emissão de gás, mas
sim o papel do homem nela, apresentando falhas, como mostrar fotos do
medidores estçao próximos a fornos, tetos de edifícios e até churrasqueiras,
e erros de algoritmos utilizados na simulações. Também expõe a natureza,
dizendo que as mudanças climáticas são fenômenos naturais e que nós
liberamos menos CO2 que os insetos. Mas o que importa para ele é essa causa
ambiental estar sendo usada como uma nova relação política para manter o
poder mundial, onde se troca antigas relações comerciais por outras,
travestidas de ecológicas. Com isso, grandes países estariam impondo acordos
comerciais ou produtos que beneficiam suas indústrias detentoras de
tecnologias ''limpas''. Para concluir, ele acha que é necessário o Brasil
ficar de olho nos novos acordos para escapar desse novo imperialismo.

O biólogo Fernando Reinach, em seu artigo ~\cite{Fernando}, defende o gás
carbônico, dizendo que quem põe a culpa na emissão de CO2 no aquecimento
global é desinformado ou manipulado, pois qualquer ser vivo na Terra emite
ou consume CO2 para sua existência e acabar com a emissão seria acabar com a
vida no planeta. Também explica o que aconteceu com o DNA, que em
campanhas contra a biotecnologia o DNA foi tão difamado que várias pessoas
não queriam comer alimentos que continham DNA, sendo que todo dia comemos
enormes quantidades de DNA. Na verdade substituíram o medo de agrotóxicos
pelo medo de DNA. Para ele, a culpa não está na molécula, e sim na queima de
combustível fóssil 
